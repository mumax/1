Micromagnetic/Spin-\/Lattice Simuations on the GPU

The core library (trunk/core) contains the building blocks for running magnetic simulations on the GPU. The high-\/level building blocks provide:
\begin{DoxyItemize}
\item convolutions (\hyperlink{gpuconv1_8h}{gpuconv1.h}, ...)
\item time stepping (\hyperlink{structgpueuler_a498c5070bf962e7189b40e06fb7cab17}{gpueuler.h}, \hyperlink{structgpuheun_a46f8656e846de7e3db0842175856eb78}{gpuheun.h}, ...)
\item micromagnetic kernels
\item unit conversion (units.h)
\item tensor utilities (\hyperlink{tensor_8h}{tensor.h})
\item performance measurment (\hyperlink{timer_8h}{timer.h})
\end{DoxyItemize}

These building blocks are used by the main simulation programs in trunk/app

Some lower-\/level functions that are used by the above building blocks include:
\begin{DoxyItemize}
\item FFT's (\hyperlink{gputil_8h}{gputil.h})
\item GPU data manipulation/communication (\hyperlink{gputil_8h}{gputil.h})
\item communication with other processes (\hyperlink{pipes_8h}{pipes.h})
\item ...
\end{DoxyItemize}

These are usually not directly needed by a main simulation program.

Aditionally, some auxilary programs are present in trunk/app. Their binaries are put in trunk/bin, which should be added to your \$PATH.

These programs include:
\begin{DoxyItemize}
\item tensor, for manipulating or post-\/processing data in the tensor format (see \hyperlink{tensor_8h}{tensor.h})
\item kernel, for generating micromagnetic kernels
\item config, for generating initial magnetic configurations
\item ...
\end{DoxyItemize}

These command-\/line programs can be handy for the user, or can be directly called from a main simulation program (e.g., using \hyperlink{pipes_8h}{pipes.h}).

\begin{DoxyAuthor}{Author}
Arne Vansteenkiste 

Ben Van de Wiele 
\end{DoxyAuthor}
