\section{Running}

\subsection{Launching the program}

To run \prog, type the following code in a terminal:

\shell{\prog \textit{arguments} \textit{input.in}}

Where \textit{\file{input.in}} is the name of the input file to run. It does not necessarily have to end with ``\file{.in}'' but we use this convention throughout this manual. The \textit{\cmd{arguments}} are optinal and described further on. 

Upon running an input file, an output directory with a corresponding name but ending with ``\file{.out}'' will be created. This directory will contain the simulation output as well as a file called \idxfile{output.log}. All the output that 

A \prog input file specifies an entire simulation, including which output should be saved. Each file contains a series of \idx{commands} which are executed in the order they are provided. Let us start with a simple example:


\begin{verbatim}
# material
msat:       	800E3 
aexch:      	1.3E-11
alpha:      	0.02
# geometry 
size:       	1     32          128    
cellsize:   	3E-9  3.90625E-9  3.90625E-9  # m
# initial magnetization
uniform		1 1 1
# run
autosave:	table	ascii	10E-12
run:          	1E-9
\end{verbatim}


\subsection{Program arguments}\label{arguments}

To obtain a list of program arguments, run \prog \cmd{-help}.
