\section{The \prog daemon}

When you run \prog \idxcmd{-daemon} \cmd{directory1} \cmd{directory2} \cmd{...}, \prog will run in \idx{daemon} mode: it will search the directories \emph{directory1 directory2 ...} for input files to run. This provides a way to sequentially run a large batch of simulations or to distribute simulations on a cluster or cloud --- as described below.

\subsection{Batch execution}

When you run \prog \idxcmd{-daemon} \cmd{directory}, all input files in the directory will be executed sequentially. The program will run all files ending with \file{.in}, for which the corresponding output directory (ending with \file{.out}) does not yet exist.  When all simulations are finished, the directory will be re-checked for new input files every few seconds. So you can add new input files to the directory at any moment and they will queued for execution.

Note that the existence of the \file{.out} directory signals that the corresponding simulation is running or finished. 

\subsection{Simulating on a cluster}

\subsection{Simulating in a cloud}
